\documentclass[a4paper,12pt]{article}
\usepackage{ upgreek }
\usepackage{ tipa }
\usepackage[T2A]{fontenc}
\usepackage[utf8]{inputenc}
\usepackage[english,russian]{babel}


\usepackage{amsmath,amsfonts,amssymb,amsthm,mathtools}


\begin{document}

\subsection*{22.3.Интегралы вида $\int\sin{\upalpha x} \: \cos{\upbeta x} dx$}

Интегралы
\[\int\sin{\upalpha x} \cos{\upbeta x} dx, \quad \int\sin{\upalpha x} \sin{\upbeta x} dx \quad \text{и} \quad \int\cos{\upalpha x} \cos{\upbeta x} dx\]
непосредственно вычисляются, если их подынтегральные функции преобразовать по формулам
\[\sin{\upalpha x} \cos{\upbeta x}=\frac{1}{2}[\sin{(\upalpha + \upbeta) x} + \sin{(\upalpha - \upbeta) x}] ,\]
\[\sin{\upalpha x} \sin{\upbeta x}=\frac{1}{2}[\cos{(\upalpha - \upbeta) x} - \cos{(\upalpha + \upbeta) x}] ,\]
\[\cos{\upalpha x} \cos{\upbeta x}=\frac{1}{2}[\cos{(\upalpha + \upbeta) x} + \cos{(\upalpha - \upbeta) x}] .\]
Например,
\[\int\sin{2x}\cos{x} dx = \frac{1}{2}\int(\sin{3x}+\sin{x}) dx= \]
\[=-\frac{1}{6}\cos{3x}-\frac{1}{2}\cos{x}+C.\]

\subsection*{22.4.Интегрвлы от трансцендентных функций, \\ вычисляющиеся с помощью \\ интегрирования по частям}

К таким интегралам относятся, например, интегралы
\[\int e^{\upalpha x}\cos{\upbeta x} \: dx,\int e^{\upalpha x}\sin{\upbeta x} \: dx,\int x^{n}\cos{\upalpha x} \: dx,\int x^{n}\sin{\upalpha x} \: dx,\]
\[\int x^n e^{\upalpha x} \: dx, \int x^n \arcsin{x} \: dx, \int x^n \arccos{x} \: dx, \int x^n \arctg{x} \: dx,\]
\[ \int x^n \arcctg{x} \: dx, \int x^n \ln{x} \: dx \; (n-\text{целое неотрицательное}).\]
Все эти интегралы вычисляются с помощью, вообще говоря,  последовательного интегрирования по частям. Действительно, имеем

\[I=\int e^{\upalpha x}\cos{\upbeta x} \: dx=\int e^{\upalpha x}d\frac{\sin{\upbeta x}}{\upbeta}=\]
\[=\frac{e^{\upalpha x}\sin{\upbeta x}}{\upbeta}-\frac{\upalpha}{\upbeta}\int e^{\upalpha x}\sin{\upbeta x} \:dx=\frac{e^{\upalpha x}\sin{\upbeta x}}{\upbeta}-\frac{\upalpha}{\upbeta}\int e^{\upalpha x}d(-\frac{\cos{\upbeta x}}{\upbeta})=\]
\[=\frac{e^{\upalpha x}\sin{\upbeta x}}{\upbeta}+\frac{\upalpha e^{\upalpha x}\sin{\upbeta x}}{\upbeta^2}-\frac{\upalpha^2}{\upbeta^2}\int e^{\upalpha x}\cos{\upbeta x} \: dx =\]
\[=\frac{e^{\upalpha x}(\upbeta\sin{\upbeta x}+\upalpha\cos{\upbeta x})}{\upbeta^2} - \frac{\upalpha^2}{\upbeta^2}I, \]
откуда
\[I=\frac{e^{\upalpha x}(\upbeta\sin{\upbeta x}+\upalpha\cos{\upbeta x})}{\upalpha^2 + \upbeta^2} + C. \eqno (22.3)\]

Аналогично интегралу $\int e^{\upalpha x}\cos{\upbeta x} \: dx$ вычисляется интеграл
\[\int e^{\upalpha x}\sin{\upbeta x} \: dx = \frac{e^{\upalpha x}(\upalpha\sin{\upbeta x}+\upbeta\cos{\upbeta x})}{\upalpha^2 + \upbeta^2},\]
а через эти два интеграла легко выражаются интегралы
\[\int\sh{\upalpha x}\cos{\upbeta x} \: dx, \int\sh{\upalpha}\sin{\upbeta x} \: dx,\]
\[\int\ch{\upalpha x}\cos{\upbeta x} \: dx, \int\ch{\upalpha}\sin{\upbeta x} \: dx.\]
Впрочем, эти последние четыре интеграла можно вычислить и непосредственно с помощью интегрирования по частям подобно тому, как был вычислен рассмотренный выше интеграл (22.3).

В интегралах $\int x^n \cos{\upalpha} \:dx, \; \int x^n \sin{\upalpha} \:dx, \; \int x^n e^{\upalpha x} \:dx,$ положив $u = x^n$ и соответственно $dv = cos{\upalpha x} \: dx, \; dv = \sin{\upalpha x} \: dx, \; dv =  e^{\upalpha x} \:dx,$ после интегрирования по частям снова придем к интегралу одного из указанных видов, но уже с показателем степени у переменной $x,$ меньшим на единицу. Применяя этот прием $n$ раз, придем к интегралу рассматриваемого типа с $n = 0,$ который, очевидно, сразу берется. Например,

\[\int x^2\sin x \: dx= \int x^2 \: d(-\cos x)= -x^2\cos x + 2\int x\cos x \: dx=\]
\[=-x^2\cos x + 2\int x \: d\sin x= -x^2\cos x + 2x\sin x - 2\int\sin x \: dx=\]
\[= -x^2\cos x + 2x\sin x + 2\cos x + C.\]

Используя интегралы, рассмотренные выше, можно вычислить и более сложные интегралы. Вычислим,
например, интеграл $\int x^n e^{\upalpha x} \cos{\upbeta x} \:dx.$

Интегрируя по частям и применяя (22.3), имеем

\[\int x^n e^{\upalpha x} \cos{\upbeta x} \: dx= \int x^n \: \left[\frac{e^{\upalpha x}(\upbeta\sin{\upbeta x}+\upalpha\cos{\upbeta x})}{\upalpha^2 + \upbeta^2}\right] =\]
\[=x^n e^{\upalpha x} \frac{\upbeta\sin{\upbeta x}+\upalpha\cos{\upbeta x}}{\upalpha^2 + \upbeta^2} - \frac{n \upbeta}{\upalpha^2 + \upbeta^2} \int x^{n - 1} e^{\upalpha x} \sin{\upbeta x} \:dx -\]
\[ - \frac{n \upalpha}{\upalpha^2 + \upbeta^2} \int x^{n - 1} e^{\upalpha x} \cos{\upbeta x} \:dx.\]

\end{document}